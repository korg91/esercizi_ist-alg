\documentclass[12pt,a4paper]{report}

\usepackage{amsmath}
\usepackage{bbm}
\usepackage[utf8]{inputenc}
\usepackage[italian]{babel}
\usepackage{longtable}
\usepackage{amsthm}
\usepackage{amscd}
\usepackage{amssymb}
\usepackage{amsfonts}
\usepackage{amsmath}
\usepackage{mathtools}
\usepackage{enumitem}
\usepackage[scale=3]{ccicons}  % per le icone creative commons
\usepackage{hyperref}  % per i link nel pdf
\usepackage[rmargin=3.0cm,lmargin=3.0cm]{geometry}
\usepackage{setspace}  % per l'interlinea
\usepackage[italian]{babel}  % per sillabazione



\theoremstyle{definition}
\newtheorem{teo}{Teorema}[section]  % resetta la numerazione dei teoremi per ogni capitolo
\newtheorem{defn}[teo]{Definizione}  % la numerazione delle definizioni dipende da quella dei teoremi
\newtheorem{es}[teo]{Esempio}  % idem
\newtheorem{oss}[teo]{Osservazione}  % idem
\newtheorem{prop}[teo]{Proposizione}  % idem
\newtheorem{lemma}[teo]{Lemma}  % idem
\newtheorem{corollario}[teo]{Corollario}  % idem



\DeclareMathOperator{\dom}{dom}
\DeclareMathOperator{\aaa}{\textit{A}^{\star}}
\DeclareMathOperator{\a01}{\{0,1\}^{\star}}
\DeclareMathOperator{\ec}{\textit{EXPCOM}}
\DeclareMathOperator{\ex}{\textit{EXP}}
\DeclareMathOperator{\car}{char}
\DeclareMathOperator{\id}{id}
\DeclareMathOperator{\gal}{Gal}


\newcommand{\cupdot}{\mathbin{\mathaccent\cdot\cup}}



% Interlinea 1.5
%\onehalfspacing  


\begin{document}

\noindent Andrea Gadotti

\section*{Capitolo 6}
\
\\
\noindent\textbf{Esercizio 1}\\
\\
Per tutto questo esercizio, indico con $K$ il campo di spezzamento del polinomio in questione e con $G$ il gruppo di Galois richiesto.
\begin{itemize}
\item[(a)] Il polinomio $x^3-x-1$ è irriducibile perché è di grado 3 e né 1 né -1 sono radici. Il suo discriminante è $\Delta=4-27=-23$, che non è un quadrato in $\mathbb{Q}$. Quindi $G \simeq S_3$ (cfr. pag. 270, Lang).
\item[(b)] Osservo che $x^3-1=(x-1)(x^2+x+1)=(x-1)(x-\varepsilon)(x-\varepsilon^2)$ dove $\varepsilon$ è una radice terza dell'unità diversa da 1. Quindi $K$ è il campo di spezzamento del polinomio separabile $x^2+x+1$. Perciò $|G|=2$. È chiaro che i due omomorfismi sono $\sigma_1, \sigma_2$ dati da $\sigma_1=\id$ e $\sigma_2(\varepsilon)=\varepsilon^2$.
\item[(c)] L'elemento $\varepsilon$ dell'esercizio (b) è $\varepsilon=\frac{-1 \pm \sqrt{-3}}{2}$. È evidente che $\varepsilon \not\in \mathbb{Q}(\sqrt{2})$, quindi anche in questo caso il polinomio $x^2+x+1$ è irriducibile e perciò $|G|=2$ e quindi $G$ è lo stesso del punto (b).
\item[(d)] Guardando al punto (c), è immediato che $\varepsilon \in \mathbb{Q}(\sqrt{-3})$, quindi il polinomio si spezza completamente su di esso. Quindi $K=\mathbb{Q}(\sqrt{-3})$ e $G=\{\id\}$.
\item[(e)] Si può vedere che il polinomio $x^3-x-1$ è irriducibile anche su $\mathbb{Q}(\sqrt{-23})$ (trovando esplicitamente le radici). Come abbiamo già osservato nel punto (a), il discriminante è $-23$, che è un quadrato in $\mathbb{Q}(\sqrt{-23})$. Quindi $G \simeq A_3$.
\item[(f$'$)] $f(x)=x^4-5=(x-\alpha)(x+\alpha)(x-i\alpha)(x+i\alpha)$ con $\alpha=\sqrt[4]{5}$. Quindi il campo di spezzamento di $f$ è $\mathbb{Q}(i,\alpha)$, che è banalmente un'estensione di Galois di $\mathbb{Q}$ di grado 8. Osservo che esiste $\tau \in \gal(K/\mathbb{Q}(\alpha))$ determinato da $\tau(i)=-i$. Inoltre $\tau^2=\id$. D'altra parte, esiste anche $\sigma \in \gal(K/\mathbb{Q}(i))$ determinato da $\sigma(\alpha)=i\alpha$. Inoltre si può verificare facilmente che $\id, \sigma, \sigma^2, \sigma^3$ sono distinti e che $\sigma^4=\id$. Poiché $\tau$ non coincide con nessuno di questi quattro automorfismi, e osservando (si vede facilmente) che $\tau\sigma=\sigma^3\tau$, trovo che 
$$G=<\tau, \sigma \mid \tau\sigma=\sigma^3\tau>$$
ovvero $G \simeq D_4$, che infatti ha 8 elementi, come cercato.
\item[(f$''$)] Sia $F:=\mathbb{Q}(\sqrt{5})$. È chiaro che $[K:F]=4$. Con riferimento al punto (f$'$), osservo che esiste $\tau \in \gal(K/F(\alpha))$ determinato da $\tau(i)=-i$. Inoltre $\tau^2=\id$. D'altra parte, esiste anche $\sigma \in \gal(K/F(i))$ determinato da $\sigma(\alpha)=-\alpha$, ed è chiaro che $\gal(K/F(i))=\{\id, \sigma\}$, in quanto ogni suo automorfismo deve mandare $\alpha^2=\sqrt{5}$ in $\alpha^2$.
Si verifica facilmente che $\tau\sigma \in G$ e che $\tau\sigma \neq \tau, \sigma$. Quindi, poiché $|G|=4$, si trova $G=\{\id, \tau, \sigma, \tau\sigma\}$.
\item[(m)]  $f(x)=x^n-t=(x-\alpha)(x-\varepsilon\alpha)(x-\varepsilon^2\alpha)...(x-\varepsilon^{n-1}\alpha)$ dove $\varepsilon$ è una radice $n$-esima primitiva di 1 e $\alpha^n=t$. Quindi $K=\mathbb{C}(t)(\alpha, \varepsilon)=\mathbb{C}(\alpha)$. Mostro che $f(x)$ è irriducibile: consideriamo l'anello $\mathbb{C}[t]$. Poiché $t$ è trascendente su $\mathbb{C}$, è chiaro che $(t)$ è massimale, e quindi primo, in $\mathbb{C}[t]$. Quindi, per il criterio di Eisenstein, $f(x)$ è irriducibile su $\mathbb{C}[t]$. Per il lemma di Gauss, $f(x)$ è irriducibile anche sul suo campo delle frazioni, ovvero su $\mathbb{C}(t)$.\\
Quindi $|G|=n$, ed è chiaro quindi che $G=\langle \sigma \rangle$ dove $\sigma(\alpha)=\varepsilon\alpha$ (e quindi $\sigma^n=\id$).
\item[(n)] $f(x)=x^4-t=(x-\alpha)(x-i\alpha)(x+\alpha)(x+i\alpha)$, dove $\alpha^4=t$. Quindi $K=\mathbb{R}(t)(i,\alpha)=\mathbb{R}(i,\alpha)$, che ha banalmente grado 4 su $\mathbb{C}(t)$ e quindi 8 su $\mathbb{R}(t)$. Ragionando come nel punto (f$'$), si trova che $G \simeq D_4$.
\end{itemize}\ 
\\
\noindent\textbf{Esercizio 3}\\
\\
Grazie al lemma di Gauss, per verificare che i polinomi in questione sono irriducibili su $\mathbb{C}(t)$, è sufficiente verificare che lo siano su $\mathbb{C}[t]$. Ed essendo polinomi di terzo grado, si verifica facilmente che tutti i polinomi dell'esercizio sono irriducibili in quanto non hanno soluzioni in $\mathbb{C}[t]$ (basta guardare i divisori del termine noto). A questo punto, possiamo applicare come in precedenza il ragionamento che sfrutta il discriminante del polinomio. In particolare, si verifica subito che per tutti i polinomi dell'esercizio, tranne l'ultimo, il discriminante non è un quadrato in $\mathbb{C}(t)$. Calcoliamo ora il discriminante dell'ultimo polinomio $x^3+t^2x-t^3$:
$$\Delta=-4(t^2)^3-27(-t^3)^2=-31t^6$$
che è il quadrato di $i\sqrt{31}t^3 \in \mathbb{C}(t)$.





 











\end{document}
