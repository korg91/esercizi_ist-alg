\documentclass[12pt,a4paper]{report}

\usepackage{amsmath}
\usepackage{bbm}
\usepackage[utf8]{inputenc}
\usepackage[italian]{babel}
\usepackage{longtable}
\usepackage{amsthm}
\usepackage{amscd}
\usepackage{amssymb}
\usepackage{amsfonts}
\usepackage{amsmath}
\usepackage{mathtools}
\usepackage{enumitem}
\usepackage[scale=3]{ccicons}  % per le icone creative commons
\usepackage{hyperref}  % per i link nel pdf
\usepackage[rmargin=3.0cm,lmargin=3.0cm]{geometry}
\usepackage{setspace}  % per l'interlinea
\usepackage[italian]{babel}  % per sillabazione



\theoremstyle{definition}
\newtheorem{teo}{Teorema}[section]  % resetta la numerazione dei teoremi per ogni capitolo
\newtheorem{defn}[teo]{Definizione}  % la numerazione delle definizioni dipende da quella dei teoremi
\newtheorem{es}[teo]{Esempio}  % idem
\newtheorem{oss}[teo]{Osservazione}  % idem
\newtheorem{prop}[teo]{Proposizione}  % idem
\newtheorem{lemma}[teo]{Lemma}  % idem
\newtheorem{corollario}[teo]{Corollario}  % idem



\DeclareMathOperator{\dom}{dom}
\DeclareMathOperator{\aaa}{\textit{A}^{\star}}
\DeclareMathOperator{\a01}{\{0,1\}^{\star}}
\DeclareMathOperator{\ec}{\textit{EXPCOM}}
\DeclareMathOperator{\ex}{\textit{EXP}}
\DeclareMathOperator{\car}{char}


\newcommand{\cupdot}{\mathbin{\mathaccent\cdot\cup}}



% Interlinea 1.5
%\onehalfspacing  


\begin{document}

\noindent Andrea Gadotti

\subsection*{Esercizio 1}\ \\
Vogliamo esprimere $(\alpha^2+\alpha+1)(\alpha^2+\alpha)$ e $(\alpha-1)^{-1}$ nella forma
$$a \alpha^2 + b \alpha +c$$
Poiché $\alpha^3+\alpha^2+\alpha+2=0$, si ha $\alpha^3=-\alpha^2-\alpha-2$, e quindi $\alpha^4=\alpha \alpha^3=-\alpha^3-\alpha^2-2\alpha=-\alpha+2$. Allora:
$$(\alpha^2+\alpha+1)(\alpha^2+\alpha)=\alpha^4+2\alpha^3+2\alpha^2+\alpha=-2\alpha-2$$
Quindi $a=0, b=-2, c=-2$.\\
Consideriamo ora $\beta:=(\alpha-1)^{-1}$. $\beta$ è tale che $(\alpha-1)\beta=1$, ovvero $(a \alpha^2 + b \alpha +c)(\alpha-1)=1$. Sviluppando il prodotto si trova:
$$(b-2a)\alpha^2+(c-a-b)\alpha-(2a+c)=1$$
Abbiamo quindi il seguente sistema:
$$\left\{
\begin{array}{l}
b-2a=0 \\
c-a-b=0 \\
2a+c=0 
\end{array}
\right.
$$
Si trova facilmente che il sistema ha soluzione $a=-\frac{1}{5}, b=-\frac{2}{5}, c=-\frac{3}{5}$.
\\
\\
\\
\noindent\textbf{Esercizio 2}\\
\\
Supponiamo $[F(\alpha):F]=2n+1$ per qualche $n \in \mathbb{N}$. Allora:
$$[F(\alpha):F(\alpha^2)][F(\alpha^2):F]=[F(\alpha):F]=2n+1$$
quindi entrambi i fattori del prodotto sono dispari. Affermiamo che $[F(\alpha):F(\alpha^2)]=1$ (e quindi $F(\alpha)=F(\alpha^2)$). Osserviamo che il polinomio $x^2-\alpha^2 \in F(\alpha^2)[x]$ ha banalmente $\alpha$ come radice, quindi $[F(\alpha):F(\alpha^2)]$ è 1 o 2. Ma 2 non è dispari, quindi $[F(\alpha):F(\alpha^2)]=1$.
\\
\\
\\
\noindent\textbf{Esercizio 3}\\
\\
Abbiamo
$$[F(\alpha, \beta):F(\alpha)][F(\alpha):F]=[F(\alpha, \beta):F]=[F(\alpha, \beta):F(\beta)][F(\beta):F]$$
ovvero
$$[F(\alpha, \beta):F(\alpha)] \cdot \deg(f)=[F(\alpha, \beta):F(\beta)] \cdot \deg(g)$$
Quindi $\deg(g) \, | \, [F(\alpha, \beta):F(\alpha)] \cdot \deg(f)$. Poiché $\deg(f)$ e $\deg(g)$ sono coprimi per ipotesi, si ha $\deg(g) \, | \, [F(\alpha, \beta):F(\alpha)]$. Ma ovviamente $[F(\alpha)(\beta):F(\alpha)] \leq [F(\beta):F]=\deg(g)$, quindi $[F(\alpha, \beta):F(\alpha)]=\deg(g)$, ovvero $g$ è irriducibile su $F(\alpha)$.
\\
\\
\\
\noindent\textbf{Esercizio 4}\\
\\
Sia $\alpha:=\sqrt[4]{2}$. Il suo polinomio minimo su $\mathbb{Q}$ è banalmente $x^4-2$, quindi $\mid \mathbb{Q}(\alpha):\mathbb{Q} \mid=4$. Stiamo cercando i campi intermedi. Supponiamo allora di avere $\beta \in \mathbb{Q}(\alpha)$ con $\beta \not\in \mathbb{Q}$. Poiché
$$\mid \mathbb{Q}(\alpha)(\beta):\mathbb{Q}(\beta) \mid \cdot \mid \mathbb{Q}(\beta):\mathbb{Q} \mid =4$$
è chiaro che dobbiamo avere $\mid \mathbb{Q}(\beta):\mathbb{Q} \mid=2$.\\
Consideriamo allora il polinomio minimo di $\beta$ su $\mathbb{Q}$. Esso dovrà essere della forma $x^2+\gamma+\delta$. Supponiamo $\beta$ e $\beta'$ radici. Allora $\beta+\beta'=\gamma \in \mathbb{Q}$ e $\beta \beta'=\delta \in \mathbb{Q}$. Poiché
$$\mathbb{Q}(\sqrt[4]{2})=\{a+b\alpha+c\alpha^2+d\alpha^3 \mid \alpha^4=2, \; a,b,c,d \in \mathbb{Q}\}$$
si ha che 
$$\beta=a+b\alpha+c\alpha^2+d\alpha^3, \; \; \beta'=a'+b'\alpha+c'\alpha^2+d'\alpha^3$$
dove i coefficienti stanno in $\mathbb{Q}$. Poiché $\beta+\beta' \in \mathbb{Q}$, si ha banalmente $b=-b', c=-c', d=-d'$. Quindi
$$\beta=a+ \xi \alpha, \; \; \beta'=a'-\xi \alpha, \; \text{ con } \xi:=b+c\alpha+d\alpha^2$$
Osservando che $\beta\beta'=(a+\xi\alpha)(a'-\xi\alpha) \in \mathbb{Q}$ e facendo i conti si trova facilmente che $b=0$ e $d=0$. Ovvero $\beta=a+c\alpha^2$, ovvero $Q(\beta)=\mathbb{Q}(\sqrt{2})$.
\\
\\
\\
\noindent\textbf{Esercizio 5}\\
\\
Sia $f(x)=x^6+x^3+1 \in \mathbb{Q}[x]$. Osserviamo che $(x^3-1)f(x)=x^9-1$, quindi le radici di $f$ in $\mathbb{C}$ sono le radici none dell'unità che non sono anche radici terze. Quindi, poiché $\sigma(\alpha)$ deve essere ancora una radice di $f$ (e l'omomorfismo è completamente determinato da essa), abbiamo che gli omomorfismi cercati sono 6 e sono dati da:
$$\sigma_k(\alpha):= e^{2\pi i k/9}$$
dove $k=1,2,4,5,7,8$.
\\
\\
\\
\noindent\textbf{Esercizio 6}\\
\\
$\sqrt{2}+\sqrt{3}$ è algebrico su $\mathbb{Q}$, di grado 4. Infatti $\sqrt{2}+\sqrt{3}$ è radice del polinomio
$$x^4-10x^2+1$$
Rimane da mostrare che $[\mathbb{Q}(\sqrt{2}+\sqrt{3}):\mathbb{Q}]=4$.
Osserviamo innanzitutto che $[\mathbb{Q}(\sqrt{2},\sqrt{3}):\mathbb{Q}]=4$. Questo si mostra facilmente con la formula dei gradi, in quanto è semplice far vedere che $\sqrt{3} \not\in \mathbb{Q}(\sqrt{2})$.\\
Ma allora, per la formula dei gradi, abbiamo che
$$[\mathbb{Q}(\sqrt{2},\sqrt{3}):\mathbb{Q}(\sqrt{2}+\sqrt{3})] [\mathbb{Q}(\sqrt{2}+\sqrt{3}):\mathbb{Q}]=4$$
Mostriamo che $[\mathbb{Q}(\sqrt{2},\sqrt{3}):\mathbb{Q}(\sqrt{2}+\sqrt{3})]=1$. Infatti $(\sqrt{2}+\sqrt{3})^3=11\sqrt{2}+9\sqrt{3}$, quindi 
$$\frac{(\sqrt{2}+\sqrt{3})^3-9(\sqrt{2}+\sqrt{3})}{2}=\sqrt{2}$$
Quindi $[\mathbb{Q}(\sqrt{2},\sqrt{3}):\mathbb{Q}(\sqrt{2}+\sqrt{3})]=1$ e perciò $[\mathbb{Q}(\sqrt{2}+\sqrt{3}):\mathbb{Q}]=4$.
\\
\\
\\
\noindent\textbf{Esercizio 7}\\
\\
Sappiamo che qualsiasi estensione di grado finito è un'estensione algebrica. Poiché $E$ e $F$ hanno grado finito, abbiamo che $E=k(\alpha_1,...,\alpha_n)$ e $F=k(\beta_1,...,\beta_m)$. Quindi $EF=k(\alpha_1,...,\alpha_n,\beta_1,...,\beta_m)$. Per la formula dei gradi, sappiamo che
$$[EF:k]=[EF:F][F:k]$$
Poiché ogni polinomio in $k[x]$ è anche un polinomio in $F[x]$, è chiaro che $[EF:F] \leq [E:k]$, ovvero
$$[EF:k] \leq [E:k][F:k]$$
Mostriamo che se $[E:k]$ e $[F:k]$ sono coprimi, allora vale l'uguaglianza. Ricordiamo che $E=k(\alpha_1,...,\alpha_n)$.  Procediamo per induzione su $n$:
\begin{itemize}
\item Se $n=0$, allora significa che $EF=F$, quindi la tesi è banalmente soddisfatta.
\item Se $n>0$, supponiamo $[k(\alpha_1,...,\alpha_n)F:F] < [k(\alpha_1,...,\alpha_n):k]$. Per la formula dei gradi,
$$[k(\alpha_1,...,\alpha_n)F:F]=[(k(\alpha_1,...,\alpha_{n-1})F)(\alpha_n):k(\alpha_1,...,\alpha_{n-1})F][k(\alpha_1,...,\alpha_{n-1})F:kF]$$
e
$$[k(\alpha_1,...,\alpha_n):k]=[k(\alpha_1,...,\alpha_{n-1})(\alpha_n):k(\alpha_1,...,\alpha_{n-1})][k(\alpha_1,...,\alpha_{n-1}):k]$$
Poiché per ipotesi il primo termine è strettamente minore del secondo, significa che
\begin{itemize}
\item[(a)] $[k(\alpha_1,...,\alpha_{n-1})F:kF] < [k(\alpha_1,...,\alpha_{n-1}):k]$, oppure
\item[(b)] $[(k(\alpha_1,...,\alpha_{n-1})F)(\alpha_n):k(\alpha_1,...,\alpha_{n-1})F] < [k(\alpha_1,...,\alpha_{n-1})(\alpha_n):k(\alpha_1,...,\alpha_{n-1})]$
\end{itemize}

Se vale (a), per ipotesi induttiva significa che $[k(\alpha_1,...,\alpha_{n-1}):k]$ e $[F:k]$ non sono coprimi. Ma allora neanche $[E:k]=[k(\alpha_1,...,\alpha_{n-1})(\alpha_n):k(\alpha_1,...,\alpha_{n-1})][k(\alpha_1,...,\alpha_{n-1}):k]$ e $[F:k]$ sono coprimi, e quindi la dimostrazione è conclusa.\\
Se non vale (a), allora deve valere (b). Ma allora possiamo proseguire ``estraendo'' ogni volta un $\alpha_j$, e prima o poi dovremo necessariamente giungere a una situazione sostanzialmente uguale ad (a).
\end{itemize}\ 
\\
\\
\noindent\textbf{Esercizio 8}\\
\\
Procediamo per induzione su $n$. Se $n=1$ la tesi è banalmente vera. Supponiamo ora che sia vero per ogni numero $\leq n$ e mostriamo che vale anche per $n+1$. Sia allora $F$ un campo e sia $f(x) \in F[x]$ un polinomio di grado $n+1$. Sia $K$ il campo di spezzamento di $f$. 
\begin{itemize}
\item Se $f(x)$ è riducibile, sia $p(x) \in F[x]$ un suo fattore irriducibile (quindi $1 \leq \deg p \leq n)$. Sia $E$ il campo di spezzamento di $p(x)$. Allora $f(x)=p(x)g(x)$ per qualche $g(x) \in E[x]$. Abbiamo allora che $K$ è il campo di spezzamento di $g(x)$ su $E$. Per ipotesi induttiva, otteniamo  $[E:F] \, | \, (\deg p)!$ e $[K:E] \, | \, (\deg g)!$. Per la formula dei gradi sappiamo che $[K:F]=[K:E][E:F]$. Quindi
$$[K:F] \, | \, (\deg p)! \, (\deg g)!$$
Ma poiché $\forall a,b \in \mathbb{N}$ vale $a! \, b! \, | \, (a+b)!$, abbiamo che $[K:F] \, | \, (\deg p + \deg g)!=(\deg f)!$ e la tesi è provata.
\item Se $f(x)$ è irriducibile, sia $\alpha$ una sua radice. Chiaramente $[F(\alpha):F]=n+1$ e $f(x)=(x-\alpha)g(x)$ con $g(x) \in F(\alpha)$ che ha grado $n$. Quindi $K$ è il campo di spezzamento di $g(x)$ su $F(\alpha)$ e allora per ipotesi induttiva $[K:F(\alpha)] \, | \, n!$. Per la formula dei gradi vale
$$[K:F]=[K:F(\alpha)][F(\alpha):F]$$
e perciò $[K:F] \, | \, (n+1)n!=(n+1)!$.
\end{itemize}
\ 
\\
\\
\noindent\textbf{Esercizio 9}\\
\\
Sia $f(x)=x^{p^8}-1 \in \mathbb{Z}/p\mathbb{Z}[x]$ con $p$ primo. Allora $f(x)=(x-1)^{p^8}$, quindi l'unica radice di $f(x)$ è 1. Perciò in campo di spezzamento di $f$ è $\mathbb{Z}/p\mathbb{Z}[x]$ stesso.
\\
\\
\\
\noindent\textbf{Esercizio 10}\\
\\
Sia $\alpha \in \mathbb{R}$ tale che $\alpha^4=5$.
\begin{itemize}
\item[(a)] È chiaro che $\alpha^2=\sqrt{5}$. Vogliamo mostrare che $\mathbb{Q}(i\alpha^2)=\mathbb{Q}(i\sqrt{5})$ è estensione normale di $\mathbb{Q}$. Sia $f(x)=x^2+5$. Le radici di $f$ sono $\pm i\sqrt{5} \in \mathbb{Q}(i\sqrt{5})$, che quindi è il campo di spezzamento per il polinomio $f$.
\item[(b)] Osserviamo che $(\alpha+i\alpha)^2=\alpha^2+2i\alpha^2-\alpha^2=2i\alpha^2 \in \mathbb{Q}(i\alpha^2)$. Quindi $\mathbb{Q}(\alpha+i\alpha)$ è il campo di spezzamento per il polinomio $f(x)=x^2-2i\alpha^2 \in \mathbb{Q}(i\alpha^2)$.
\item[(c)] Osserviamo che $\alpha+i\alpha$ è radice del polinomio $f(x)=x^4+20 \in \mathbb{Q}[x]$. Inoltre $f$ è irriducibile, in quanto in $\mathbb{C}$ abbiamo che 
$$f(x)=(x-\beta_1)(x-\beta_2)(x-\beta_3)(x-\beta_4)$$
dove $\beta_1=\alpha+i\alpha, \beta_2=-\alpha+i\alpha, \beta_3=-\alpha-i\alpha, \beta_4=\alpha-i\alpha$, e nessuno dei fattori sta in $Q[x]$ e nemmeno nessun prodotto di due fattori.\\
Ci basta quindi mostrare che esiste una radice di $f$ che non sta in $\mathbb{Q}(\alpha+i\alpha)$.\\
Supponiamo per assurdo che $\beta_1, \beta_4 \in \mathbb{Q}(\alpha+i\alpha)$. Allora anche $\alpha=\beta_1-\beta_4 \in \mathbb{Q}(\alpha+i\alpha)$, e quindi anche $i=\beta_1 \alpha^{-1} \in \mathbb{Q}(\alpha+i\alpha)$. Quindi $\mathbb{Q}(\alpha,i) \subseteq \mathbb{Q}(\alpha+i\alpha)$. Ma $\mathbb{Q}(\alpha+i\alpha)$ ha grado 4 su $\mathbb{Q}$, mentre è facile vedere che $\mathbb{Q}(\alpha,i)$ ha grado 8 su $\mathbb{Q}$ (si usa la formula dei gradi e il fatto immediato che $\sqrt[4]{5} \not\in \mathbb{Q}(i)$).
\end{itemize}
\ 
\\
\\
\noindent\textbf{Esercizio 11}\\
\begin{itemize}
\item[(c)] Su $\mathbb{C}$, $f(x)=(x-\alpha)(x-\alpha_2)(x-\alpha_3)$ dove $\alpha=\sqrt[3]{2}$, $\alpha_2=\alpha (-1/2+i\sqrt{3}/2)$, $\alpha_3=\alpha (-1/2-i\sqrt{3}/2)$. Quindi il campo di spezzamento di $f(x)$ su $\mathbb{Q}$ è $\mathbb{Q}(\alpha,\alpha_2,\alpha_3)$. Osserviamo che $\alpha_3=-\alpha-\alpha_2$, quindi $\mathbb{Q}(\alpha,\alpha_2,\alpha_3)=\mathbb{Q}(\alpha,\alpha_2)$. Quindi, preso $\varepsilon:=i\sqrt{3}$, abbiamo banalmente $\mathbb{Q}(\alpha,\alpha_2)=\mathbb{Q}(\alpha,\varepsilon)$. Affermo che $[\mathbb{Q}(\alpha,\varepsilon):\mathbb{Q}]=6$, e questo chiaro usando l'Esercizio 7 e il fatto che $[\mathbb{Q}(\sqrt[3]{2}):\mathbb{Q}]=3$ e $[\mathbb{Q}(i\sqrt{3}):\mathbb{Q}]=2$.
\end{itemize}
\ 
\\
\\
\noindent\textbf{Esercizio 12}\\
\\
Vogliamo mostrare che in $K$ valgono $\forall y \exists x (y=x^p)$ e $\forall a,b (a^p=b^p \Rightarrow a=b)$. Il secondo punto è immediato, perché $a^p-b^p=0 \Rightarrow (a-b)^p=0 \Rightarrow a-b=0 \Rightarrow a=b$. Mostriamo il primo punto: se $y=0$ è banale. Se $y\neq 0$, poiché $K$ è un campo con $p^n$ elementi, si ha che $|K^*|=p^n-1$. Quindi $y^{p^n-1}=1$, ovvero $y^{p^n}=y$. Osserviamo adesso che, preso $y^{p^{n-1}} \in K$, si ha
$$\left(y^{p^{n-1}}\right)^p=y^{p^n}=y$$
ovvero $y^{p^{n-1}}$ è la radice $p$-sima cercata.
\\
\\
\\
\noindent\textbf{Esercizio 13}\\
\\
Sia $L$ il campo di spezzamento di $f(x) \in K[x]$. Sia $A$ l'insieme di tutte le radici di $f$, che per ipotesi sono distinte e formano un campo. $A$ è quindi un campo finito, ovvero $|A|=p^n$ per qualche $n \geq 1$, dove $p=\car A$. Quindi $\deg f=p^n$. Osserviamo che 
$$g(x):=x^{p^n}-x \in K[x]$$
ha come radici proprio l'insieme $A$. Infatti $x^{p^n}-x=x(x^{p^n-1}-1)$. Quindi $g$ ha come radici 0 e tutti gli $x$ tali che $x^{p^n-1}=1$. Ma quest'ultima equazione è soddisfatta da qualsiasi elemento di $A^*$. Ne segue banalmente che $f=g$.\\
Poiché $A$ è un campo contenuto in $L$, si ha che $\car L=\car A=p=\car K$.
\\
\\
\\
\noindent\textbf{Esercizio 14}\\
\\
Sappiamo che se $\car K=p$ e $L \supseteq K$ con $[L:K]<\infty$, allora 
$$[L:K]=p^m \, [L:K]_S$$
con $m \in \mathbb{N}$. Quindi, poiché $[L:K]$ e $p$ sono coprimi per ipotesi, abbiamo che $m=0$, ovvero $[L:K]=[L:K]_S=[L \cap K^S : K]$. E poiché $L \cap K^S \subseteq L$, si ha $L \cap K^S = L$. Quindi ogni elemento di $L$ è separabile su $K$.
\\
\\
\\
\noindent\textbf{Esercizio 15}
\\
%\begin{lemma}
%Sia $K$ un campo e $\alpha$ algebrico su $K$. Sia $f(x) \in K[x]$ %tale che su $K(\alpha)[x]$ vale $f(x)=(g(x))^n$. Sia $m$ tale che %$(g(x))^m$ appartiene a $K[x]$ ed è irriducibile. Allora $m \mid n$.
%\begin{proof}
%Supponiamo per assurdo $m \nmid n$. Allora $n=mq+r$ per qualche %$0<r<m$, ovvero 
%$$f(x)=(g(x))^{mq+r}=(g(x)^m)^q (g(x))^r$$
%Quindi $(g(x))^r$ sta in $K[x]$, quindi $(g(x))^m \mid (g(x))^r$, ma %questo è assurdo perché $r<m$.
%\end{proof}
%\end{lemma}
\\
Osserviamo che
$$f(x):=x^{p^n}-a=x^{p^n}-\alpha^{p^n}=(x-\alpha)^{p^n}$$
dove $\alpha$ è una radice $p^n$-esima di $a$ in qualche estensione di $K$.\\
Sia $(x-\alpha)^r$ un fattore irriducibile di $f$ in $K[x]$. Possiamo scrivere $r=s p^m$ con $m \leq n$ e $s$ coprimo con $p$. Allora
$$(x-\alpha)^r=(x-\alpha)^{s p^m}=(x^{p^m}-\alpha^{p^m})^s= x^{sp^m}-s\alpha^{p^m} x^{(s-1)p^m}+ \dots$$
Quindi $s\alpha^{p^m} \in K$, e poiché in $K$ vale $s \neq 0$, $s$ è invertibile e perciò $\alpha^{p^m} \in K$. Quindi $(\alpha^{p^m})^{p^{n-m-1}} \in K$ è una radice $p$-esima di $a$, il che contraddice l'ipotesi.
\\
\\
\\
\noindent\textbf{Esercizio 16}\\
\\
Proviamo ``$\Longrightarrow$''. Sia $x^{p^n}-\alpha^{p^n} \in K(\alpha^{p^n})[x]$. Se per assurdo fosse irriducibile, allora esso sarebbe il polinomio minimo di $\alpha$. Ma su $K(\alpha)[x]$ vale $x^{p^n}-\alpha^{p^n}=(x-\alpha)^{p^n}$, ovvero $\alpha$ è una radice multipla del suo polinomio minimo, il che è impossibile perché $\alpha$ è separabile per ipotesi. Quindi $x^{p^n}-\alpha^{p^n}$ è riducibile, perciò grazie all'Esercizio 15 abbiamo che $K(\alpha^{p^n})$ contiene una radice $p$-esima di $\alpha^{p^n}$. La radice in questione è proprio $\alpha^{p^{n-1}}$, in quanto l'endomorfismo di Frobenius è iniettivo.\\
Quindi su $K(\alpha^{p^n})=K(\alpha^{p^{n-1}})$, e poiché quanto provato vale per ogni $n$, si ha la tesi.\\
\\
Proviamo ``$\Longleftarrow$''. Sia $f(x) \in K[x]$ il polinomio minimo di $\alpha$. Poiché $f$ è irriducibile, sappiamo che $f(x)=g(x^{p^r})$ per qualche $r \in \mathbb{N}$, dove $g(y)$ è un polinomio in $K[x]$ irriducibile e separabile. Osserviamo che $g(\alpha^{p^r})=f(\alpha)=0$, ovvero $\alpha^{p^r}$ è radice di $g$, ovvero $\alpha^{p^r}$ è separabile su $K$. Quindi $K(\alpha^{p^r})$ è un'estensione separabile di $K$. Ma $K(\alpha^{p^r})=K(\alpha)$ per ipotesi, perciò la dimostrazione è conclusa.
\\
\\
\\
\noindent\textbf{Esercizio 17}\\
\\
Proviamo $(a) \Longrightarrow (b)$. Supponiamo per assurdo che esista $a \in K$ che non ha una radice $p$-esima in $K$. Consideriamo il polinomio $x^p-a$ e sia $\alpha$ una sua radice. Nel suo campo di spezzamento, abbiamo che
$$x^p-a=x^p-\alpha^p=(x-\alpha)^p$$
Questo significa che il polinomio minimo di $\alpha$ su $K$ divide $(x-\alpha)^p$, ovvero il polinomio in questione ha radici multiple, ovvero $K(\alpha)$ non è separabile, e questo contraddice l'ipotesi.\\
Proviamo  $(b) \Longrightarrow (a)$. Supponiamo per assurdo che esista un $\alpha$ algebrico su $K$ il cui polinomio minimo $f$ possiede radici multiple (possiamo anche supporre senza perdità di generalità che $\alpha$ sia proprio una radice multipla del suo polinomio minimo). Sappiamo che questo significa che $\car K=p>0$ e che $f$ è della forma
$$f = x^{pn} + a_{n-1}x^{p(n-1)} + \dots + a_1 x^p + a_0$$ 
Per ipotesi, possiamo trovare $b_0,...,b_{n-1}$ tali che 
$$f = x^{pn} + b_{n-1}^px^{p(n-1)} + \dots + b_1^p x^p + b_0^p$$ 
ovvero
$$f = (x^{n} + b_{n-1}x^{n-1} + \dots + b_1 x + b_0)^p$$ 
Poiché $f(\alpha)=0$, allora anche $(\alpha^{n} + b_{n-1} \alpha^{n-1} + \dots + b_1 \alpha + b_0)^p=0$, e poiché l'endomorfismo di Frobenius è iniettivo, questo significa che $\alpha^{n} + b_{n-1} \alpha^{n-1} + \dots + b_1 \alpha + b_0=0$, ovvero
$$x^{n} + b_{n-1}x^{n-1} + \dots + b_1 x + b_0 \in K[x]$$
ha $\alpha$ come radice ed ha grado strettamente minore di $f$, quindi abbiamo una contraddizione.
\\
\\
\\
\noindent\textbf{Esercizio 18}\\
\\
Vogliamo dimostrare che se $K$ è un campo finito, allora $\forall x \in K, \; x=a^2+b^2$ per qualche $a,b \in K$.
\begin{itemize}
\item Se $\car K=2$, allora $x^{2n-1}=1$, ovvero $x=x^{2n}=(x^n)^2$. Quindi $x=(x^n)^2+0^2$.
\item Se $\car K=p>2$, consideriamo il sottogruppo moltiplicativo $K^*$. $K^*$ è un gruppo di ordine $p^n-1$ per qualche $n$ ed è ciclico. Quindi gli elementi che si possono scrivere come quadrati sono almeno tutte le potenze pari del generatore di $K^*$, ovvero sono almeno $(p^n-1)/2$. Poiché $0=0^2$, i quadrati in $K$ sono almeno $(p^n+1)/2$. Quindi, se $A:=\{y \in K \; | \; y=a^2 \text{ per qualche } a \in K \}$ e $X:=\{x \in K \; | \; x=a^2+b^2 \text{ per qualche } a,b \in K \}$, abbiamo che $X=A+A$. Si vede facilmente che se $S$ è un sottoinsieme di un gruppo finito $G$, e vale $|S|+|S|>|G|$, allora $G=S+S$. Poiché $|A|+|A|=p^n+1>p^n=|K|$, si ha che $K=A+A=X$.
\end{itemize}
\ 
\\
\noindent\textbf{Esercizio 19}\\
\\
Sia $E \subseteq R \subseteq F$ un sottoanello di $E$. Sia $\alpha \in R$. È sufficiente mostrare che $\alpha^{-1} \in R$. Osserviamo che, poiché $\alpha$ è algebrico su $F$, esiste $f(x)= \sum_{i=0}^n a_i x^i$ irriducibile tale che $f(\alpha)= \sum_{i=0}^n a_i \alpha^i = 0$ ($a_0 \neq 0$ perché $f(x)$ è irriducibile).\\
Quindi $\alpha (-\sum_{i=1}^n a_i/a_0 \; \alpha^{i-1})=1$ dove $-\sum_{i=1}^n a_i/a_0 \; \alpha^{i-1} \in R$, e si ha la tesi.\\
Mostriamo che l'ipotesi di algebricità per $E$ è necessaria, esibendo un controesempio: sia $F=\mathbb{Q}$ e sia $E=\mathbb{Q}(\pi)$. Sia $R$ il più piccolo anello contenente $Q$ e $\pi$. È chiaro che 
$$R=\left\{ \sum_{i=1}^n c_i \pi^i \; \mid \; c_i \in \mathbb{Q}, \; n \in \mathbb{N} \right\}$$
e quindi $\frac{1}{\pi} \not\in R$.
\\
\\
\\
\noindent\textbf{Esercizio 20}

\begin{itemize}
\item[a)] Sia $y \in K$. Poiché $y \in F(x)$, si ha che $y=\frac{f(x)}{g(x)}$ per qualche $f, g \in F[t]$. È chiaro che $f(t)-g(t) y=f(t)-g(t) \frac{f(x)}{g(x)} \in K[t]$ ha $x$ come radice.
\item[b)] Sia $p(t)=f(t)-g(t) y=f(t)-g(t) \frac{f(x)}{g(x)} \in F(y)[t]$. È chiaro che $p(t)$ ha $x$ come radice e $\deg p=\max(\deg f, \deg g)$. Vogliamo mostrare che $p(t)$ è irriducibile. Iniziamo enunciando il seguente:

\begin{lemma}
Siano $f,g \in F[t]$ coprimi. Siano $P_n \in F[t]$ polinomi di grado strettamente minore di $m := \max(\deg(f), \deg(g))$. Se 
$$\sum_{n=0}^N P_n f^n g^{N-n} = 0$$ 
per qualche $N \in \mathbb{N}$, allora $P_n=0$ per ogni $n$.
\end{lemma}
\begin{proof}
Assumiamo senza perdità di generalità che $\deg(g) = m$. Allora $g$ divide $\sum_{n=0}^{N-1} P_n f^n g^{N-n}$, perciò divide anche $P_N f^N$. Per coprimalità, abbiamo che $g$ divide $P_N$, Ma dall'ipotesi sul grado dei $P_n$ segue che $P_N=0$. Otteniamo quindi che $\sum_{n=0}^{N-1} P_n f^n g^{(N-1)-n} = 0$ e si prosegue per induzione.
\end{proof}

Sia ora $y = \frac{f(x)}{g(x)}$, con $f$ e $g$ coprimi in $F[t]$. Vogliamo mostrare che il polinomio minimo di $x$ in $F(y)$ ha grado $m = \max(\deg(f), \deg(g))$. Sia $p$ un polinomio in $F(y)[t]$, tale che $p(x) = 0$. Scriviamo $p(t) = \sum_{k=0}^r a_k t^k$, con $a_k \in F(y)$ e supponiamo $r < m$. Possiamo scrivere $a_k = \frac{P_{k}(y)}{Q_{k}(y)}$, con $P_{k}, Q_{k} \in F[t]$. Vale allora

$$\sum_{k=0}^r \frac{P_{k}(y)}{Q_{k}(y)} x^k = 0$$

Quindi otteniamo

$$\sum_{k=0}^r P_{k}(y) \prod_{k' \neq k} Q_{k'} (y) x^k = 0,$$

che riscriviamo come
$$\sum_{k=0}^r \widetilde{P_{k}(y)} x^k = 0,$$

dove $\widetilde{P_{k}} := P_{k}(y) \prod_{k' \neq k} Q_{k'} (y)$.

Se definiamo adesso $\widetilde{P_{k}} =: \sum_l a_{kl} t^l \in F[t]$, otteniamo 

$$
\sum_{k=0}^r \sum_l a_{kl} \frac{f(x)^l}{g(x)^l} x^k = 0.
$$

Moltiplichiamo ora tutto per $g(x)^L$ con $L$ abbastanza grande. Troviamo

$$
\sum_{k=0}^r \sum_l a_{kl} f(x)^l g(x)^{L - l} x^k = 0,
$$ 

Che possiamo riscrivere come

$$
\sum_{l} (\sum_{k= 0}^r a_{kl} x^k) f(x)^l g(x)^{L-l} = 0.
$$

Grazie al lemma (e alla trascendenza di $x$), tutti gli $a_{kl}$ sono 0. Quindi anche gli $\widetilde{P_{k}}$ sono 0 e si vede subito che anche tutti gli $a_k$ sono 0. In conclusione, $p$ è il polinomio nullo, e questo conclude la dimostrazione.
\end{itemize}




 











\end{document}
